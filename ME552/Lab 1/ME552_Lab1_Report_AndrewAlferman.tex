%%%%%%%%%%%%%%%%%%%%%%%%%%%%%%%%%%%%%%%%%
% Short Sectioned Assignment
% LaTeX Template
% Version 1.0 (5/5/12)
%
% This template has been downloaded from:
% http://www.LaTeXTemplates.com
%
% Original author:
% Frits Wenneker (http://www.howtotex.com)
%
% License:
% CC BY-NC-SA 3.0 (http://creativecommons.org/licenses/by-nc-sa/3.0/)
%
%%%%%%%%%%%%%%%%%%%%%%%%%%%%%%%%%%%%%%%%%

%----------------------------------------------------------------------------------------
%	PACKAGES AND OTHER DOCUMENT CONFIGURATIONS
%----------------------------------------------------------------------------------------

\documentclass[paper=letter, fontsize=10pt]{scrartcl} % A4 paper and 11pt font size

\usepackage[T1]{fontenc} % Use 8-bit encoding that has 256 glyphs
\usepackage{fourier} % Use the Adobe Utopia font for the document - comment this line to return to the LaTeX default
\usepackage[english]{babel} % English language/hyphenation
\usepackage{amsmath,amsfonts,amsthm} % Math packages
\usepackage{graphicx}
\usepackage{caption}
\usepackage[titletoc,title]{appendix}
\usepackage{titlesec}
\usepackage{footnote}
\usepackage{sectsty} % Allows customizing section commands
\allsectionsfont{\centering \normalfont\scshape} % Make all sections centered, the default font and small caps

\makesavenoteenv{tabular}

\usepackage{fancyhdr} % Custom headers and footers
\pagestyle{fancyplain} % Makes all pages in the document conform to the custom headers and footers
\fancyhead{} % No page header - if you want one, create it in the same way as the footers below
\fancyfoot[L]{} % Empty left footer
\fancyfoot[C]{} % Empty center footer
\fancyfoot[R]{\thepage} % Page numbering for right footer
\renewcommand{\headrulewidth}{0pt} % Remove header underlines
\renewcommand{\footrulewidth}{0pt} % Remove footer underlines
\setlength{\headheight}{13.6pt} % Customize the height of the header

%\numberwithin{equation}{section} % Number equations within sections (i.e. 1.1, 1.2, 2.1, 2.2 instead of 1, 2, 3, 4)
%\numberwithin{figure}{section} % Number figures within sections (i.e. 1.1, 1.2, 2.1, 2.2 instead of 1, 2, 3, 4)
%\numberwithin{table}{section} % Number tables within sections (i.e. 1.1, 1.2, 2.1, 2.2 instead of 1, 2, 3, 4)

% \setlength\parindent{0pt} % Removes all indentation from paragraphs - comment this line for an assignment with lots of text

%----------------------------------------------------------------------------------------
%	TITLE SECTION
%----------------------------------------------------------------------------------------

\newcommand{\horrule}[1]{\rule{\linewidth}{#1}} % Create horizontal rule command with 1 argument of height

\title{	
\normalfont \normalsize 
\textsc{ME 552} \\ [25pt] % Your university, school and/or department name(s)
\horrule{0.5pt} \\[0.4cm] % Thin top horizontal rule
\huge Report for Lab 1: Calibration of Volumetric Flow Measurement Devices \\ % The assignment title
\horrule{2pt} \\[0.5cm] % Thick bottom horizontal rule
}

\author{Andrew Alferman} % Your name

\date{\normalsize\today} % Today's date or a custom date

\begin{document}

\iffalse
%THINGS HERE ARE THE OBJECTIVES TO HIT ON FROM THE LAB 1 FILE
To include in report:

Part 1: 	Plots of the calibration for both the orifice plates and the rotameters.

Part 2:	The value reported by the Isco-pumps as the independent variable and the calibrated flow rate as the dependent variable.

Part 3:	Plot data for MKS inputs to actual flow rates determined by Gilibrator.  Be sure to show uncertainty bars (design state uncertainty on the measured and dependent variables.


Questions to answer:

Part 1:	1) How does the discharge coefficient compare to that reported for the orifice plate?  Is the discharge coefficient constant for compressible and incompressible flow?
		2) Are compressibility effects a concern for the rotameter over the range that you studied? Justify your answer.

Part 2:	1) Are the values reported by the Isco-pumps appropriate to use, or does a calibration need to be applied to the values reported for the Isco-pumps.  Consider if values fall within the uncertainty bounds.

Part 3:	1) Discuss why the calibrated and MKS reported values are linearly related or not

\fi

\maketitle % Print the title

\section{Introduction}
\label{intro}
Three separate volumetric flow measurement experiments were conducted using a variety of common sensors and flow control equipment.  In the first experiment, air was directed through an orifice plate, a rotameter, and a dry test meter installed in series.  The orifice plate and rotameter were both calibrated using pressure data and measurements taken from the dry test meter.  The volumetric flow rate of water from an Isco-pump was calibrated in the second experiment using a calibrated scale and a stopwatch.  The volumetric flow rate of air through a MKS controller was measured in the third experiment using a bubble Gilibrator device.  The data obtained from the experiments was analyzed to determine the impact of compressibility effects and an uncertainty analysis was performed.  The objectives of these experiments were to familiarize the student with calibrating and using volumetric flow rate measurement equipment and to obtain flow rate data that will be used in a later experiment using a controlled flame.

\section{Methodology}
\subsection{Experiment Description}
The first experiment directed filtered shop air through an orifice, a rotameter, and a dry test meter that were installed in series.  A diagram of the test setup can be found % ADD A DIAGRAM IN AN APPENDIX
After opening the supply valve to admit air to the test setup, the flow rate was throttled using a valve in the rotameter until the rotameter read each one of the values in % NEED A REFERENCE TO A TABLE HERE
A data acquisition system was started and the pressure drop across the orifice was measured using pressure transducers located immediately upstream and downstream of the orifice.  Additionally, the temperature of the air was measured at % NEED THE POINT AT WHICH TEMPERATURE WAS MEASURED
\newline % Need to figure out a better way of doing this, eventually.
\newline
The volumetric flow rate of water through an Isco-pump was measured in the second experiment using a calibrated scale and a stopwatch.  An empty container was placed on the scale and the pump outlet was positioned to discharge into the container.  After zeroing the scale with the container on it, the pump was started at the same time that a stopwatch was started.  After 90 seconds, the pump was stopped and the weight of the container with water was recorded.
\newline
\newline
The third experiment 

\subsection{Equipment Used}
A list of all of the equations used to perform the required calibrations and volumetric flow rates as well as an uncertainty analysis of the data collected can be found in Appendix \ref{app:Equip}

\subsection{Analysis}
The methodology used to determine the 

\section{Assumptions}
To make calibration of equipment possible, at least one measurement device in each experiment was assumed to be calibrated prior to collecting data.  These measurement devices were the pressure transducers and the dry test meter in the first experiment, the scale in the second experiment, and the bubble Gilibrator in the third experiment.
Information regarding the tolerance of each measurement device can be found in % ADD APPENDIX THING HERE.
The manufacturer specifications listed in (APPENDIX THING) are assumed to be correct for all measurement devices used in this lab.
\newline
\newline
The assumed condition in which compressibility effects are significant is as follows:

\begin{equation}
\frac{P_1 - P_2}{P_1} \geq 0.1
\end{equation}

in which \(P_1\) is the upstream pressure and \(P_2\) is the downstream pressure, as measured by the pressure transducers.

The following fluid properties were used for each of the experiments:

% ADD A TABLE OF ALL OF THE FLUID PROPERTIES

\section{Results}
The flow rate in the first experiment blah blah answer the question

We found that water is lighter than air in the second experiment.

We found that human sacrifices are necessary in the third experiment.

\section{Conclusion}
% CONCLUSION GOES HERE

\clearpage
\begin{appendices}
\setcounter{equation}{0}
% "INCLUDE IN THE APPENDIX DOCUMENTATION ABOUT HOW THE DESIGN STAGE UNCERTAINTY WAS DETERMINED, YOUR UNCERTAINTY TREES, AND ANYTHING THAT IS NEEDED TO SUPPORT YOUR DISCUSSION."
\section{Equipment Tolerances}\label{app:Equip}

\begin{center}
\begin{tabular}{ | c | c | c |}
 \hline
 Instrument & Model & Data \\
 \hline\hline
 Orifice & OModel & Data \\
 \hline
 Rotameter & RModel & Data \\
 \hline
 Pressure Transducer & PTModel & Data \\
 \hline
 Dry Test Meter & Singer DTM-200\footnote{The American Meter division of Singer was acquired through a series of transactions by Elster American Meter, therefore specifications on the Elster American Meter DTM-200 were assumed to be identical to the Singer DTM-200.} & Data \\
 \hline
 Isco-pump & IPModel & Data \\
 \hline
 Scale & Smodel & Data \\
 \hline
 Bubble Gilibrator & BGModel & Data \\
 \hline
 MKS Thermal Flow Controller & FCModel & Data \\
 \hline
\end{tabular}
\end{center}

\clearpage
\section{Uncertainty Analysis}
\subsection{Methodology}
%Here's where I write my thrilling description of what I did for this analysis, including equations.

The equation for subsonic flow through an orifice is as follows:

\begin{equation}\label{eq:1}
Q = C E A Y \sqrt{\frac{2 \Delta P}{\rho_1}}
\end{equation}

in which Q is the volumetric flow rate, C is the orifice discharge coefficient, A is the area of the orifice opening, Y is a compressibility constant, \(\Delta P\) is the pressure difference across the orifice, \(\rho_1\) is the density of the fluid upstream of the orifice (air in this case).  E is defined as follows:

\begin{equation}\label{eq:2}
E =  \frac{1}{\sqrt{1 - (\frac{A_0}{A_1})^2}}
\end{equation}

in which \(A_0\) and \(A_1\) are the area of the orifice opening and the area of the upstream piping, respectively.  The orifice in the first experiment is calibrated by solving for C in equation \ref{eq:1} of this appendix.  If it is instead found that choked flow conditions exist (i.e. the flow is sonic), the mass flow rate through the orifice is calculated using the following equation:

\begin{equation}\label{eq:3}
\dot{m}_{choked} = C \frac{P_0 A_1}{\sqrt{R T_0}}\sqrt{\frac{2k}{k+1}\Bigg(\frac{2}{k+1}\Bigg)^\frac{2}{k+1}}
\end{equation}

in which \(\dot{m}_{choked}\) is the mass flow rate of the fluid, \textit{k} is the ratio of specific heats upstream and downstream of the orifice% ADD THE OTHER VARIABLES AS NEED BE


\subsection{Tree Diagram}
%A tree diagram goes here

\subsection{Results}

\clearpage
\section{Raw Data}

\end{appendices}
\end{document}